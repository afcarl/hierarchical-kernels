% \renewcommand{\subparagraph}{\paragraph}
% \usepackage{include/icml2014}
\usepackage{natbib}
\usepackage{amsmath, amsthm, amssymb}
% \usepackage{pstricks}
% \usepackage{pst-tree}
%\usepackage{color}
%\usepackage{makeidx}  % allows for indexgeneration
\usepackage{bm}
%\usepackage{graphicx}
\usepackage{epsfig}
\usepackage{caption}
\usepackage{subcaption}
%\usepackage{floatrow}
\usepackage{amsmath}
\usepackage{amsfonts}
\usepackage{color}
\usepackage{array}
\usepackage{colortbl}
\usepackage{framed}
\usepackage{url}
\usepackage{booktabs}
\usepackage{multirow}
%\usepackage{sgame}
\usepackage{dsfont}
\def\sgtextcolor{black}
\def\sglinecolor{black}
%\renewcommand{\gamestretch}{2}
\usepackage{multicol}
\usepackage{lscape}
\usepackage{relsize}
\usepackage{rotating}
\usepackage{tikz}
\usetikzlibrary{calc}
% \usepackage{titlesec}


% ============== Jasper Spacing Hackery =========
%\titlespacing{\section}{0pt}{0pt}{0pt}
%\titlespacing{\subsection}{0pt}{0pt}{0pt}
%\titlespacing{\subsubsection}{0pt}{0pt}{0pt}
%\titlespacing{\paragraph}{0pt}{0pt}{10pt}
% \usepackage{enumitem}
% \setlist[itemize]{noitemsep, topsep=0pt} % Makes els 1 and 2 fit more nicely
% ===========================================

% ==============Kevin's commands ===============
%\newfloatcommand{capbtabbox}{table}[][\FBwidth]

% ============== Mike's commands ==============
\usepackage{nicefrac}
%\newcommand{\vect}[1]{\underline{\smash{#1}}}
%\newcommand{\vect}[1]{{#1}}
\renewcommand{\vec}[1]{\mathbf{#1}}
\renewcommand{\v}[1]{\vec{#1}}
\newcommand{\reals}{\mathds{R}}
\newcommand{\sX}{\mathcal{X}}
\newcommand{\sD}{\mathcal{D}}
\newcommand{\br}{}%{^{\text{\textnormal{ r}}}}
\newcommand{\cat}{^\textnormal{c}}
% ============== ==============

\newcommand{\cut}[1]{}
\newcommand{\hide}[1]{}
% \renewcommand{\blue}[1]{{\textcolor{blue}{#1}}}
%\renewcommand{\blue}[1]{#1}


\newcommand\transpose{{\textnormal{\tiny{\sf{T}}}}}
\newcommand{\hlinespace}{~\vspace*{-0.15cm}~\\\hline\\\vspace*{0.15cm}}
%\newcommand{\hlinespace}{~\vspace*{0.45cm}\\\hline\\~\vspace*{-0.9cm}}
%\newcommand{\hlinespace}{~\vspace*{0.05cm}\\\hline~\vspace*{0.5cm}}

% comment the next line to turn off notes
%\renewcommand{\note}[1]{~\\\frame{\begin{minipage}[c]{\textwidth}\vspace{2pt}\center{#1}\vspace{2pt}\end{minipage}}\vspace{3pt}\\}

\newcommand{\lnote}[1]{\note{#1}}
\newcommand{\emcite}[1]{\citet{#1}}
%\newcommand{\yrcite}[1]{\citeyear{#1}}
\newcommand{\aunpcite}[1]{\citeR{#1}}

\newcommand{\heavyrule}{\specialrule{\heavyrulewidth}{.4em}{.4em}}
\newcommand{\lightrule}{\specialrule{.03em}{.4em}{.4em}}

%%%%%%%%%%%%%%%%%%%%%%%%%%%%%%%%%%%%%%%%%%%%%%

%% keep figures from going onto a page by themselves
\renewcommand{\topfraction}{0.9}
\renewcommand{\textfraction}{0.07}
\renewcommand{\floatpagefraction}{0.9}
\renewcommand{\dbltopfraction}{0.9}      % for double-column styles
\renewcommand{\dblfloatpagefraction}{0.7}   % for double-column styles

\newtheorem{thm}{Theorem}%[section]
\newtheorem{lem}[thm]{Lemma}
\newtheorem{prop}[thm]{Proposition}
\newtheorem{cor}[thm]{Corollary}
\newtheorem{obs}[thm]{Observation}

%\theoremstyle{definition}
\newtheorem{define}[thm]{Definition}
\hyphenation{ge-ne-ral-ize}

\newcommand{\Var}{\ensuremath\textnormal{Var}}
\newcommand{\indicator}{\ensuremath\mathds{I}}

%\newcommand{\fhspace}{\vspace*{0.2cm}}
%\newcommand{\newsec}{\hspace{0cm}}

% replaces tabular; takes same arguments. use \midrule for a rule, no vertical rules, and eg \cmidrule(l){2-3} as needed with \multicolumn
\newenvironment{ktabular}[1]{\sffamily\small\begin{center}\begin{tabular}[c]{#1}\toprule}{\bottomrule \end{tabular}\end{center}\normalsize\rmfamily\vspace{-5pt}}
\newcommand{\tbold}[1]{\textbf{#1}}
\newcommand{\interrowspace}{.6em}

\newcommand{\embeddingletter}{g}
\newcommand{\bo}{{\sc bo}}
\newcommand{\gp}{{\sc gp}}
\newcommand{\agp}{Arc \gp}

\usepackage{float}